\documentclass{article}

\usepackage{qtree}
\usepackage{amsmath}

\newcommand{\fallingfactorial}[1]{%
  ^{\underline{#1}}%
}

\title{CSCI 301 Homework 1}
\date{2018-04-22}
\author{Brandon Chavez}

\begin{document}
	\maketitle
	\pagenumbering{gobble}
	\newpage
	\pagenumbering{roman}
	
	\paragraph{Problem 1}
	
	This problem is asking for a set containing every member of the Power Set of \{1,2,3\} which contains the number 2.
	It is possible to discover every such set by simply mapping out each subset of \{1,2,3\} and cherrypicking out the
	sets which contain two, like so:
	
	\qtreecentertrue
	\Tree [.$\emptyset$ [.\{1\} [.\{1,2\} \{1,2,3\} \{1,2\} ]
	                                         [.\{1\} \{1,3\} \{1\} ] ]
				       [.$\emptyset$ [.\{2\} \{2,3\} \{2\} ] [.$\emptyset$ \{3\} $\emptyset$ ] ] ]
	
	Doing so, one can easily pick out the sets containing 2 and make a set of these elements. However, there
	exists an even simpler alternative: logically, the power set of \{1,3\} contains every set that does not include
	the number 2 in the power set of \{1,2,3\}. To illustrate for the sake of proving that statement:
	
	\qtreecentertrue
	\Tree [.$\emptyset$ [.\{1\} \{1,3\} \{1\} ] 	
			               [.$\emptyset$ \{3\} $\emptyset$ ] ]
			               
	So adding 2 to each of these sets gives us every subset of \{1,2,3\} which contains 2. So the set we are
	looking for is
	
	\begin{equation*}
		\{\textit{X} \in \textit{P} ( \,\{1,2,3\}) \,: 2 \in \textit{X}\} = \{\{1,2,3\}, \{1,2\}, \{2,3\}, \{2\}\}
	\end{equation*}		
	
	\paragraph{Problem 2}
	
	The negation of a statement of the form $\textit{P} \Rightarrow \textit{Q}$ is $\textit{P} \land \neg \textit{Q}.$
	If we allow \textit{P}, \textit{Q}, and \textit{R} represent the logical statements, ``x is a rational number," 
	and ``$x \neq 0$,"  and ``$tan( \, x) \,$ is not a rational number respectively," then it becomes clear that 
	Problem 2 presents such a statement. Specifically, $( \, \textit{P} \land \textit{Q} ) \, \Rightarrow \textit{R}$. 
	Therefore, the negation of $( \, \textit{P} \land \textit{Q} ) \, \Rightarrow \textit{R}$ is 
	$( \, \textit{P} \land\textit{Q}) \, \land \neg \textit{R}$. Translated back into English, this reads "\textit{x} is a rational number and \textit{x} does not equal zero, 
	but $tan(x)$ is a rational number.
	
	\paragraph{Problem 3}
	
	The set \{1,2,3,4,5,6,7\} contains four different odd numbers. If they are to be placed first, and 
	must be placed in a sequence, then there are four possible layouts for such an initial placement as follows:
	
	\begin{table} [h!] 
	\centering
		\begin{tabular} { | c | c | c | c | c | c | c | }
		\hline
		Odd & Odd & Odd & Odd & Even & Even & Even\\ 
		\hline
		\end{tabular}
		\caption{Case 1}
	\end{table}
	
	\newpage
	
	\begin{table} [h!] 
	\centering
		\begin{tabular} { | c | c | c | c | c | c | c | }
		\hline
		Even & Odd & Odd & Odd & Odd & Even & Even\\ 
		\hline
		\end{tabular}
		\caption{Case 2}
	\end{table}
	
	\begin{table} [h!] 
	\centering
		\begin{tabular} { | c | c | c | c | c | c | c | }
		\hline
		Even & Even & Odd & Odd & Odd & Odd & Even\\ 
		\hline
		\end{tabular}
		\caption{Case 3}
	\end{table}	
	
	\begin{table} [h!]
	\centering
		\begin{tabular} { | c | c | c | c | c | c | c | }
		\hline
		Even & Even & Even & Odd & Odd & Odd & Odd\\ 
		\hline
		\end{tabular}
		\caption{Case 4}
	\end{table}
	
	So there are is 1 choice (with 4 options) with, followed by 4 choices in which 
	we must decide what numbers will actually be placed. Starting with $n$ options for the first 
	choice, each successive number of options is $n - 1$ as we have one less option for 
	the number we shall place each time. So the number of possible lists so far is 
	$4 * 4\fallingfactorial{4}$. Extending the same logic to the placement of the remaining digits 
	(the positive ones), we find that the remaining possible choices can be 
	quantified by $3\fallingfactorial{3}$. Thus, the total number of possible lists in this scenario is
	$4*4\fallingfactorial{4}*3\fallingfactorial{3} = 120$.
	
	\paragraph{Problem 4}
	
	Note that the sets containing every 4-card hand in which all 4 cards are from different suites and 
	in which all 4 cards are red are \textit{mutually exclusive}. In other words, it is impossible for 
	a 4-card hand of all red cards to \textit{also} be a 4-card hand with one card from each suite, 
	because two of the suites are black. This means that we can simply find all possible hands for each
	scenario, add them, and the resulting number is the answer. Naturally, there can be no repetition because
	each card in a deck is unique to that deck.
	
	\begin{table} [h!]
	\centering
		\begin{tabular} { | c | c | c | c | }
		\hline
		Suit1 & Suit2 & Suit3 & Suit4 \\ 
		\hline
		\end{tabular}
		\caption{Every card is from a different suite.}
	\end{table}
	
	The first choice has 52 cards to choose from. However, every choice afterwards has 13 less cards because
	whatever suit was picked last cannot be picked again if the condition is to be satisfied. Thus, for 4 choices, 
	the total amount of possible lists is $52*39*26*13 = 685,464$.
	
	\begin{table} [h!]
	\centering
		\begin{tabular} { | c | c | c | c | }
		\hline
		redCard1 & redCard2 & redCard3 & redCard4 \\ 
		\hline
		\end{tabular}j
		\caption{Every card is red.}
	\end{table}
	
	Here, the number of possible lists is simply $26\fallingfactorial{4}$ or 358,800, as the first choice 
	has 26 cards, the second has 25, etc., since repetition is not allowed and half of the cards 
	in the deck are red. 
	
	Summing up these possible lists, the total number of lists that meet the question's criteria is:
	
	\begin{align}
	\prod_{i=0}^{n = 3} 52 - 13x + 26\fallingfactorial{4}\\
	= 1,044,264
	\end{align}

\end{document}

	